\documentclass[a4paper]{article}

%% Language and font encodings
\usepackage[english]{babel}
\usepackage[utf8x]{inputenc}
\usepackage[T1]{fontenc}

%% Sets page size and margins
\usepackage[a4paper,top=3cm,bottom=2cm,left=3cm,right=3cm,marginparwidth=1.75cm]{geometry}

%% Useful packages
\usepackage{amsmath}
\usepackage{graphicx}
\usepackage[colorinlistoftodos]{todonotes}
\usepackage[colorlinks=true, allcolors=blue]{hyperref}

\title{Bankruptcy prediction using stacking methods }
\author{ShuoshuoFan}

\begin{document}
\maketitle

\begin{abstract}
what's your Problem and  its importance in modern economics.The aim of predicting financial distress is 
to develop a predictive model that combines various econometric measures and allows to foresee a financial condition of a firm.
This paper applies stacking methods to the bankruptcy prediction problem in an attempt to suggest a new model with better explanatory power
and stability.To serve this purpose,we do something.
\end{abstract}

\section{Introduction}
\begin{itemize}
\item   State the problem - broad beginning
\item   more specific area of concern  
\item  what  we know --Previous works.research literature
\item  what we do not know ,Gap Research  
\item  our method--brief overview of Method
\item  (General research question)
\item   (Specific Hypothesis )
\item   Organization of the paper.
\end{itemize}

\section{Material and Methods}

\subsection{How to add Comments}
\subsection{How to include Figures}
\subsection{How to add Tables}
\subsection{How to write Mathematics}
\subsection{How to create Sections and Subsections}
\subsection{How to add Lists}
You can make lists with automatic numbering \dots
\begin{enumerate}
\item Like this,
\item and like this.
\end{enumerate}
\dots or bullet points \dots
\begin{itemize}
\item Like this,
\item and like this.
\end{itemize}

\subsection{How to add Citations and a References List}

\section{Experiments}
\subsection{Dataset}
\subsection{Experiment setup}
\section{Results}
\begin{itemize}
\item  how Hypothesis is addressed by Results
\item  how do the Results address the Gap?
\item  what we know and do not know
\item  limitations and strengths of current study
\item  next research steps
\item  the solution -broad issue wish to address and applications of this line of research 
\end{itemize}
\section{Discussion}

\bibliographystyle{IEEEtran}
\bibliography{bankruptcy}

\end{document}